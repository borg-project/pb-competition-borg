\section{\label{sec:subsolvers}Portfolio Composition}

Portfolio methods rely entirely on the performance of the solvers they employ,
and are possible only because of the engineering and research involved in
making those solvers effective. This version of {\tt borg-pb} considered 8
subsolvers in its model, several groups of which were multiple
parameterizations of the same solver.  Table \ref{tab:subsolvers} lists these
solvers and their authors. Note that the final policy did not necessarily use
every solver.
\begin{table*}
\begin{center}
\begin{tabular}{lll}
\toprule
{\bf Name}                       & {\bf Author(s)}\\
\midrule
{\tt clasp-1.3.3}                & Martin Gebser, Benjamin Kaufmann, and Torsten Schaub\\
{\tt sat4j-pb-v20090829}         & Daniel Le Berre and Anne Parrain\\
{\tt sat4j-pb-v20090829-cutting} & Daniel Le Berre and Anne Parrain\\
{\tt bsolo\_pb10-l1}             & Vasco Manquinho\\
{\tt bsolo\_pb10-l2}             & Vasco Manquinho\\
{\tt bsolo\_pb10-l3}             & Vasco Manquinho\\
{\tt wbo1.4a}                    & Vasco Manquinho\\
{\tt wbo1.4b}                    & Vasco Manquinho\\
\bottomrule
\end{tabular}
\caption{\label{tab:subsolvers}Subsolvers considered by {\tt
borg-pb-10.05.30}.}
\end{center}
\end{table*}

